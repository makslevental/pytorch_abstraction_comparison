\subsection{Abstraction}\label{subsec:abstraction}
What is abstraction?
In fact, there are several closely related notions of abstraction.
First there is the philosophical notion of abstraction;
Locke defines abstraction as follows (bolding ours):
\begin{displayquote}[\cite{Locke1689-LOCAEC-4}]
    The acts of the mind, wherein it exerts its power over simple ideas, are chiefly these three: \textellipsis
    The third is \textbf{separating them from all other ideas that accompany them in their real existence}: this is called abstraction \textellipsis
\end{displayquote}
Then there is mathematical abstraction;
Russell defines abstraction as follows:
\begin{displayquote}[\cite{Russell1937-RUSPOM-7}]
    This principle [of abstraction] asserts that, whenever a relation, of which there are instances, has the two properties of being symmetrical and transitive, then the relation in question is not primitive, but is analyzable into sameness of relation to some other term;
    and that this common relation is such that there is only one term at most to which a given term can be so related, though many terms may be so related to a given term.
\end{displayquote}
Intriguing as these notions of abstraction may be, they are distinctly different from the notion of abstraction in computer science;
in particular with respect to mathematical abstraction (bolding ours):
\begin{displayquote}[\cite{abstraction}]
    \textellipsis the primary product of mathematics is \textit{inference structures}, while the primary product of computer science is \textit{interaction patterns}.
    This is a crucial difference, and it shapes their use of formalism and the kind of abstraction used in the two disciplines.

    \vspace{4pt}

    \textellipsis computer science is distinguished from mathematics in the use of a kind of abstraction that computer scientists call \textit{information hiding}.
    The complexity of behaviour of modern computing devices makes the task of programming them impossible without abstraction tools that hide, but do not neglect, \textbf{details that are essential in a lower-level processing context but inessential in a [particular] software design and programming context}.
\end{displayquote}
This understanding of abstraction is widely agreed upon;
notably Abelson, Sussman, and Sussman in their much revered \textit{Structure and Interpretation of Programs}:
\begin{displayquote}[\cite{abelson1996structure}]
    We are not at that moment concerned with how the procedure computes its result, only with the fact that it computes the square.
    The details of how the square is computed can be suppressed, to be considered at a later time.
    Indeed, as far as the \code{good-enough?} procedure is concerned, \code{square} is not quite a procedure but rather an abstraction of a procedure, a so-called \textit{procedural abstraction}.
    At this level of abstraction, any procedure that computes the square is equally good.
\end{displayquote}

Thus, abstraction is the modulation of concern for details in accordance with the needs of the user and levels of abstractions are graded by the degree of the elision (bolding ours):
\begin{displayquote}[\cite{abstraction}]
    To specify nontrivial computational processes in machine language is a practical impossibility for humans, and so programming languages with higher levels of abstraction are necessary.

    \vspace{4pt}

    \textellipsis At a higher level of [abstraction], a \textit{subroutine}, \textit{function}, or \textit{procedure} is an abstraction of a segment of memory that hides the details of how the segment represents a piece of a program that is passed certain parameter values and returns a value as a result.

    \vspace{4pt}

    \textellipsis A \textit{garbage collector} is an abstraction of a special process that collects garbage and makes it once again available to the original program, hiding from that program the details of how this is done.


    \vspace{4pt}

    \textellipsis \textbf{This use of code libraries is an example of \textit{procedural abstraction}}, or the ability to execute code through the calling of named procedures that accept explicitly described parameters and return certain guaranteed results.
    It is an example of abstraction because the details of how the procedure performs its computation are hidden from the procedure's caller;
    since the caller only makes use of the procedure for its results, there is no need for it to know the internals.
\end{displayquote}


Taking \textit{information and concern encapsulation} as our operational definition of abstraction, in what sense shall we measure the costs of the abstractions employed by DL frameworks?
An immediate candidate measure of cost is the asymptotic time (or space) complexity of various operations and data structures that comprise the abstraction.
We claim that, with rare exception\footnote{One result does come to mind: Pippenger~\cite{10.1145/244795.244798} produces a program that runs in O$(n)$ on an impure LISP but which runs in $\Theta(n \log n)$ a pure LISP\@.}, asymptotic complexity is a poorly suited measure of the complexity or cost of abstractions in the sense that we here deal with.
If the abstraction is truly abstract then it bears no resemblance to the realization (recall Locke's definition of abstraction) and if the abstraction is reified then the analysis becomes completely impractical (owing to the numerous components and operations).
Even if such analysis were practicable the result would most likely be uninteresting and inconsequential for actual DL frameworks and their users.
It is well known that the constant factors in the complexity and peculiarities of those systems themselves more closely govern performance than the order terms.
For example, Quicksort, an O$\left(n^2\right)$ sorting routine, outperforms even many $\Theta(n\log n)$ sorting routines because it is more cache efficient~\cite{10.5555/1410219}.

Another way to reason about the cost of abstractions is according to the ``zero-overhead'' principle as articulated by Bjarne Stroustrup:
\begin{displayquote}[\cite{10.1007/978-3-642-28869-2_1}]
    In general, C++ implementations obey the zero-overhead principle: What you don't use, you don't pay for.
    And further: What you do use, you couldn't hand code any better.
\end{displayquote}
Therefore we make the expedient and practical assumption that what is more interesting and valuable to the DL community than asymptotics is, in fact, an empirical study of the resource efficiency of the abstractions;
namely execution time, memory usage, and GPU utilization.



\subsection{GPUs}\label{subsec:gpus}

We briefly review NVIDIA GPUs%
\footnote{A more comprehensive introduction to GPUs themselves and CUDA programming is available in~\cite{10.5555/1891996}.}
in order that the performance criteria we measure in~\cref{sec:methodology} are legible.

A GPU consists of many simple processors, called streaming multiprocessors (SMs), which are comprised by many compute \textit{cores} that run at relatively low clock speeds%
\footnote{For example, individual NVIDIA GTX-1080 Ti cores run at $\sim$1500MHz.}.
Each compute core in an SM can execute one floating-point or integer operation per clock cycle.
See ~\cref{fig:fermi_arch} for a diagram of NVIDIA's Fermi architecture, where each SM consists of 32 cores, 16 load/store (LD/ST) units, four special-function units (SFUs) which compute transcendental functions (such as $\sin$, $\cos$, $\exp$), a relatively large register file%
\footnote{For example, Intel's Haswell architecture supports 168 integer and 168 floating-point registers.}%
, and thread control logic (to be discussed in the proceeding).
Each SM has access to local memory, several cache levels, and global memory.
In the Fermi architecture (and subsequent architectures) local memory is configurable in software;
a fraction of it can be apportioned as either local memory or L1 cache (for workloads that query global memory in excess of local memory).
One final feature worth mentioning, though irrelevant for us here, is the L2 cache's atomic \code{read-modify-write} facilities;
this enables sharing data across groups of threads more efficiently than possible in conventional CPUs%
\footnote{On a CPU, atomic \code{test-and-set} instructions manage a semaphore, which itself manages access to memory (therefore incurring a cost of at least two clock cycles).}.

Such an architecture, particularly suited to maximizing throughput, necessitates a programming model distinct from that of a conventional, general purpose processor architecture.
A unit of computation deployed to a GPU is called a \textit{kernel}; kernels can be defined using NVIDIA's Compute Unified Device Architecture (CUDA) extensions to C, C++, and FORTRAN%
\footnote{In fact, CUDA compiles down to a virtual machine assembly code (by way of \code{nvcc}) for a virtual machine called the Parallel Thread Execution (PTX) virtual machine. So, in effect, it is compilers all the way down.}.
Compiled kernels are executed by many \textit{threads} in parallel, with each thread starting at the same instruction;
NVIDIA describes this addition to Flynn's taxonomy~\cite{5009071} as Single Instruction Multiple Thread (SIMT)%
\footnote{They key difference between SIMD and SIMT is that while in SIMD all vector elements in a vector instruction execute synchronously, threads in SIMT can diverge; branches are handled by predicated instructions~\cite{cuda_toolkit}.}.
The large register file enables very fast thread context switching ($\sim$25 microseconds on the Fermi architecture~\cite{Glaskowsky2009NVIDIAS}), performed by a centralized hardware thread scheduler.
Multiple threads are grouped into blocks (SMs are single tenant with respect to blocks) and blocks are grouped into \textit{grids} (grids execute a single kernel).
All threads in a block, by virtue of running on the same SM, coordinate (execute in arbitrary order, concurrently, or sequentially) and share memory.
Thread blocks are partitioned into \textit{warps} of 32 threads;
it is these warps that are dispatched by the warp scheduler (see~\cref{fig:cuda_cores}) and starting with the Fermi architecture two warps can be executed concurrently on the same SM in order to increase utilization%
\footnote{That is, one warp can occupy the compute cores while the other occupies the SFUs or Load/Store units.}.

\begin{figure}
    \centering
    \begin{subfigure}{\linewidth}
        \centering
        \includegraphics[width=\linewidth]{figures/fermi_arch.png}
        \caption{Eight (of 16) SM in the Fermi architecture (remaining 8 are symmetrically placed around the L2 cache)}
        \label{fig:top_half_ferm_arch}
    \end{subfigure}
    \\[3ex]
    \begin{subfigure}{\linewidth}
        \centering
        \includegraphics[width=\linewidth]{figures/cuda_cores.png}
        \caption{An individual Fermi SM}
        \label{fig:cuda_cores}
    \end{subfigure}
    \caption{NVIDIA Fermi Architecture~\cite{5751939}}
    \label{fig:fermi_arch}
\end{figure}


We present an example CUDA program in~\cref{fig:cuda_hello_world} to illustrate some of the artifacts of the CUDA threading model.
The premise of the program is performing an element-wise sum of two $32 \times 48$ entry matrices.
Note that all of the data weighs in at  $3 \times 32 \times 48 \times 4 = 18$ kilobytes (well within the bounds of shared memory on any one SM).
The actual work of summing is partitioned across a grid of six thread blocks, each containing $16 \times 16$ threads.
Such a partitioning means each thread can be logically responsible for exactly one sum and therefore the kernel is quite simple (see~\cref{lst:cuda_hello_world}).
Within the context of a kernel, each thread is uniquely identified by its multi-index in the thread hierarchy (\code{threadIdx} and \code{blockIdx}).
Hence, to carry out the sum, the kernel maps this multi-index to the physical address of the data%
\footnote{In CUDA C/C++ data is laid out in row-major order but this is not fixed (in CUDA FORTRAN the data is laid out in column-major order).}.
This (grid, block, thread)-to-data mapping is, in effect, the mechanism that implements the SIMT architecture.
Note that, since each block is allocated to exactly one SM, this sum will take $\left( 16 \times 16 \right) \div 16 = 16$ clock cycles on the Fermi architecture;
better throughput could be achieved by increasing the number of blocks (and therefore the number of SMs assigned work).

\begin{figure}
    \centering
    \begin{subfigure}{\linewidth}
        \centering
        \inputminted{c}{code/hello_world_kernel.c}
        \caption{CUDA code to be compiled by \mintinline{c}{nvcc}.
        Note differences \mintinline{c}{__global__} and \mintinline{c}{matrix_sum<<m,n>>} from standard C.}
        \label{lst:cuda_hello_world}
    \end{subfigure}
    \\[3ex]
    \begin{subfigure}{\linewidth}
        \centering
        \includegraphics[width=.8\linewidth]{figures/matrix_thread.png}
        \caption{Mapping from thread and block to matrix element.}
        \label{fig:matrix_thread}
    \end{subfigure}
    \caption{Canonical CUDA "Hello world" kernel (matrix addition).}
    \label{fig:cuda_hello_world}
\end{figure}


\subsection{Graph compilers and Tensors}\label{subsec:graph-compilers}

DL frameworks primarily function as graph compilers and tensor abstractions\footnote{A tensor in this context is a data structure similar to a multidimensional array that supports some useful operations (e.g. slicing, flattening, index permutation). Most DL frameworks also abstract memory layout on hardware behind this abstraction.};
They typically also include some ``quality of life'' utilities useful for the training of DL models (e.g.\ optimizers and data loaders).
PyTorch's \code{Tensor} abstraction is responsible for a great deal of the complexity and implementation overhead of the framework.
Due to the framework's broad support for hardware and data types, dynamic dispatch\footnote{Kernels live in shared-object libraries (e.g. \code{libcaffe2.so}, \code{libcaffe2\_gpu.so}) and therefore call sites of virtual functions (indirection) are resolved at runtime.} is employed to resolve methods on \code{Tensor}s (see~\cref{fig:dispatch}).
This dynamic dispatch produces deep call stacks for every single operation on a \code{Tensor} (see~\cref{fig:stacks}); it remains to be seen whether the context switching\footnote{Every function call corresponds to a stack frame allocation and register allocations. In addition indirection to far away call sites leads to poor instruction cache efficiency~\cite{10.5555/3314872.3314876}} between function contexts incurs any appreciable execution time penalty.

\begin{figure*}[h]
    \begin{tikzpicture}[>=latex]
        \tikzset{block/.style= {draw,rectangle,align=center,minimum width=2cm,minimum height=1cm}}
        \node [](code)  {\code{torch.conv2d(x,y)}};
%    \node [thick,block,right =0.5cm of code, draw=red, text=red](dynamic)   {device/layout \\ dispatch};
        \node [above right =1.25cm and 2cm of code, text=red](CUDA)   {\code{CUDA}};
        \node [below=0cm of CUDA, text=red](CPU) {\code{CPU}};
        \node [below=0cm of CUDA, text=red](CPU) {\code{CPU}};
        \node [below=0cm of CPU, text=red](FPGA) {\code{FPGA}};
        \node [below=0cm of FPGA, text=red](XLA) {\code{XLA}};
        \node [below=0cm of XLA, text=red](dots) {\vdots};
        \node [below=0cm of dots, text=red](vulkan) {\code{Vulkan}};
        \node [below=0cm of vulkan, text=red](mkldnn) {\code{MKLDNN}};
        \node [below=0cm of mkldnn, text=red](qcuda) {\code{QuantizedCUDA}};
%    \node [rectangle, right =0.5cm of code, draw=red, text=red, inner sep=0mm, fit= (cpu) (cuda),label=below right:device/layout \\ dispatch] {};
        \node[block,draw=red,dashed, text=red, inner sep=5mm, fit= (CUDA) (qcuda),label={[anchor=south,text=red]north:dynamic dispatch}](dynamic) {};

        \node [right=2cm of CPU](SparseCUDAByteType) {\code{SparseCUDAByteType::conv2d}};
        \node [below=0cm of SparseCUDAByteType](twodots) {\vdots};
        \node [below=0cm of twodots](CUDADoubleType) {\code{CUDADoubleType::conv2d}};
        \node [below=0cm of CUDADoubleType](threedots) {\vdots};
        \node [below=0cm of threedots](CUDAHalfType) {\code{CUDAHalfType::conv2d}};
        \node[block,draw=black,dashed, inner sep=5mm, fit= (SparseCUDAByteType) (CUDAHalfType),label={[anchor=south]north:static\ dispatch\ on\ data\ type}](dtype) {};

        \path[draw]
        (code.east) edge[->] (dynamic.west);
        \path[draw]
        (CUDA.east) edge[out=0, in=180,->] (SparseCUDAByteType.west);
        \path[draw]
        (CUDA.east) edge[out=0, in=180,->] (CUDADoubleType.west);
        \path[draw]
        (CUDA.east) edge[out=0, in=180,->] (CUDAHalfType.west);
    \end{tikzpicture}
    \caption{How the \code{torch.conv2d} operation on tensors \code{x}, \code{y} is implemented in PyTorch.}\label{fig:dispatch}
\end{figure*}

\begin{figure*}[h]
\centering
\begin{adjustbox}{width=0.8\textwidth}
  \centering
  \begin{tikzpicture}[%
      >=latex,
      level distance=10mm,
      inner sep=2mm,
      edge from parent path={[thick,->] ([thick]\tikzparentnode.south) -> (\tikzchildnode.north)}]
    \tikzstyle{calls}=[draw=black,thick,anchor=west]
    \tikzstyle{commoncode}=[pattern=north west lines, pattern color=green!50, draw=green!50!black]
    \tikzstyle{uncommoncode}=[pattern=north west lines, pattern color=red!50, draw=red!50!black]
    \node[] (node) at (1.2,0) {\Large PyTorch};
    \node[] (node) at (4.5,-2) {\Large LibTorch};
    \node[] (node) at (6.5,-7.2) {\Large cuDNN};
    \node[] (node) at (9.5,-5) {\Large TorchScript};
    \node[] (node) at (0,0) {}
    child[] { node[calls,uncommoncode] { \code{torch.conv2d} } edge from parent[draw=none]
        child { node[uncommoncode] {$\bigcdot$}
            child { node[uncommoncode] {$\bigcdot$} edge from parent[draw=none]
                child { node[uncommoncode] {$\bigcdot$} edge from parent[draw=none]
                    child[] { node[calls,commoncode] {\code{at::conv2d}}
                        child[grow=up, edge from parent path={[<-] ([thick]\tikzparentnode.north) -> (\tikzchildnode.south)}] { node[calls, uncommoncode] {\code{functional::detail::conv2d}}
                            child[grow via three points={one child at (-1.72,1) and two children at (0.1,-1) and (0.1,-2)}, edge from parent path={[<-] ([]\tikzparentnode.north) -> (\tikzchildnode.south)}] { node[calls, uncommoncode] {\code{Conv2dImpl::forward}}
                              }
                          }
                        child { node[commoncode] {$\bigcdot$}
                            child { node[commoncode] {$\bigcdot$} edge from parent[draw=none]
                                child { node[commoncode] {$\bigcdot$} edge from parent[draw=none]
                                    child[grow via three points={one child at (-0.5,-1) and
                                            two children at (0,1) and (5,0)}] { node[calls,commoncode] {\code{cudnnConvolutionForward}}
                                        child[edge from parent path={[<-] ([]\tikzparentnode.north) -> (\tikzchildnode.south)}] { node[calls, uncommoncode] {\code{Conv2d<float>::forward}}
                                          }
                                        child[edge from parent path={[<-] ([]\tikzparentnode.east) -> (\tikzchildnode.west)}] { node[uncommoncode] {$\bigcdot$}
                                            child[grow=up, edge from parent path={[<-] ([]\tikzparentnode.east) -> (\tikzchildnode.west)}] { node[uncommoncode] {$\bigcdot$} edge from parent[draw=none]
                                                child[grow=up, edge from parent path={[<-] ([]\tikzparentnode.east) -> (\tikzchildnode.west)}] { node[uncommoncode] {$\bigcdot$} edge from parent[draw=none]
                                                    child[grow via three points={one child at (-3.07,1) and
                                                            two children at (0,0) and (0,0)}, edge from parent path={[<-] ([]\tikzparentnode.north) -> (\tikzchildnode.south)}] { node[calls, uncommoncode] {\code{autograd::VariableType::\_convolution}}
                                                      }
                                                  }
                                              }
                                          }
                                      }
                                  }
                              }}}}}}};
    \draw[thick,|<->|]
    (0.22,-2) -- (0.22,-4) node [midway, fill=white] {30 calls};
    \draw[thick,|<->|]
    (1.9,-6) -- (1.9,-8) node [midway, fill=white] {61 calls};
    \draw[thick,|<->|]
    (10.5,-7) -- (10.5,-9) node [midway, fill=white] {14 calls};
  \end{tikzpicture}
\end{adjustbox}
      \caption["Short" caption without tikz code]{Call graphs representing the number of calls between \code{Conv2d.forward} at the level of abstraction and the ultimate execution of the convolution \code{cudnnConvolutionForward} on the GPU. \tikz{\path[fill=red!10,draw=red!50!black] (0,0) rectangle (.25cm,.25cm);}\, represents calls where the implementations diverge and \tikz{\path[pattern=north west lines, pattern color=green!50, draw=green!50!black] (0,0) rectangle (.25cm,.25cm);} represents calls where two or more implementations coincide. Note that program setup calls are omitted. These were produced by building each implementation with debug symbols and using \code{gdb} to set a breakpoint at \code{cudnnConvolutionForward}. Complete stacktraces are available on GitHub at \link{\href{https://github.com/makslevental/pytorch_abstraction_comparison/tree/main/tex/figures/stack_traces}{main/tex/stack\_traces}}.}
  \label{fig:stacks}
\end{figure*}


DL graph compilers are distinct from other dataflow compilers (such as VHDL and Verilog\footnote{Verilog and Very High Speed Integrated Circuit Hardware Description Language (VHSIC-HDL or VHDL) are specification languages for specifying circuits on field programmable gate arrays.}); in addition to keeping account of how the data streams through the compute graph, they also keep account of how the gradients of the data stream through the graph (i.e.\ the \textit{gradient-flow}).
This is called \textit{automatic differentiation} (often shortened to \textit{autodiff}).
In principle autodiff is implemented by using the rules of Newton's calculus to calculate the derivatives of primitive functions and the chain rule to calculate derivatives of compositions of primitive functions.
There are two types of autodiff: \textit{forward mode} (or \textit{forward accumulation}) and \textit{reverse mode} (or \textit{reverse accumulation})%
\footnote{Briefly, for a composition of functions $y=f(g(h(x)))$, forward mode evaluates the derivative $y'(x)$, as given by the chain rule, inside-out while reverse mode evaluates the derivative outside-in. For those familiar with functional programming, these operations correspond to \code{foldl} and \code{foldr} on the sequence of functions with $\partial_x$ as the operator.}.
Reverse mode autodiff enables the framework to effectively calculate the gradients of parameters of a neural network with respect to some relevant loss or objective function.
Note that such gradients can be \textit{back-propagated} through the neural network in order to adjust the parameters of the neural network such that it minimizes the loss\footnote{In which case, it is, in fact, the negatives of the gradients that are back-propagated.} or maximizes the objective.

Dataflow graphs (and their corresponding gradient-flow graphs) can be specified either statically, with fan-in and fan-out for all functions predetermined, or dynamically, where compositions of functions are determined ``on-the-run''.
There are advantages and disadvantages to both specification strategies.
Static specifications tightly constrain\footnote{For example, branches and loops are cumbersome to specify statically.} the intricacy of the dataflow graph but, obversely, can be leveraged to improve performance and scalability~\cite{le2019tflms,Pradelle2017PolyhedralOO}.
TensorFlow (prior to v2.0) is an example of a DL framework that compiles statically specified graphs.
Conversely, dynamic specifications can be very expressive and user friendly, including such conveniences as runtime debugging, but are much more difficult to optimize.
PyTorch is an example of a DL framework that supports dynamic specification.
Both PyTorch and TensorFlow also support just-in-time (JIT) compilation strategies (TorchScript and XLA respectively);
such JIT compilers strike a balance between fluency and scalability.
In this work we investigate TorchScript (see~\cref{sec:methodology}).

It warrants mention that, in addition to vertically integrated DL frameworks (i.e.\ specification language and hardware compiler), recently there has been work on intermediate bytecode representations for dataflow graphs that arbitrary compiler ``frontends'' can target.
The Multi-Level Intermediate Representation (MLIR)~\cite{lattner2020mlir} project has goals that include supporting dataflow graphs, optimization passes on those graphs and hardware specific optimizations%
\footnote{Interestingly enough, the project is headed by Chris Lattner who, in developing LLVM, pioneered the same ideas in general purpose programming languages.}.
Stripe~\cite{zerrell2019stripe} is a polyhedral compiler%
\footnote{A polyhedral compiler models complex programs (usually deeply nested loops) as polyhedra and then performs transformations on those polyhedra in order to produce equivalent but optimized programs~\cite{Griebl98codegeneration}.}
that aims to support general machine learning kernels, which are distinguished by their high parallelism with limited mutual dependence between iterations.
Tensor Comprehensions~\cite{vasilache2018tensor} is an intermediate specification language (rather than intermediate bytecode representation) and corresponding polyhedral compiler;
the syntax bears close resemblance to Einstein summation notation and the compiler supports operator fusion and specialization for particular data shapes.
Finally, Tensor Virtual Machine (TVM)~\cite{10.5555/3291168.3291211} is an optimizing graph compiler that automates optimization using a learning-based cost modeling method that enables it to efficiently explore the space of low-level code optimizations.
